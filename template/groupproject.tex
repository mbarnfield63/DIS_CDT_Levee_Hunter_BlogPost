%---------------------------------------------------------------------
% This file provides a skeleton UCL DIS CDT report.
% \pdfinclusioncopyfonts=1
% This command may be needed in order to get \ell in PDF plots to appear. Found in
% https://tex.stackexchange.com/questions/322010/pdflatex-glyph-undefined-symbols-disappear-from-included-pdf
%---------------------------------------------------------------------
% Specify where LaTeX style files can be found.
\newcommand*{\DISCDTLATEXPATH}{latex/}
% Use this variant if the files are in a central location, e.g. $HOME/texmf.
%---------------------------------------------------------------------

\documentclass[NOTE, disdraft=true, UKenglish]{\DISCDTLATEXPATH UCLCDTDISdoc}
% The language of the document must be set: usually UKenglish or USenglish.
% british and american also work!
% Commonly used options:
%  cdtdraft=true|false   This document is an UCL CDT DIS draft.
%  paper=a4|letter       Set paper size to A4 (default) or letter.

%--------------------------------------------------------------------- 
% Add you own definitions here (file dis-gp-defs.sty).
%\usepackage{dis-gp-defs}
%---------------------------------------------------------------------


%--------------------------------------------------------------------- 
% Files with references for use with biblatex.
% Note that biber gives an error if it finds empty bib files.
%\usepackage{biblatex} % uncomment if use addbibresource command
%\addbibresource{dis-gp.bib}
%--------------------------------------------------------------------- 

% Paths for figures - do not forget the / at the end of the directory name.
\graphicspath{{logos/}{figures/}}


%---------------------------------------------------------------------
% Generic document information
%---------------------------------------------------------------------
% Title, abstract and document 
%-------------------------------------------------------------------------
% This file contains the title, author and abstract.
% It also contains all relevant document numbers if needed.
%-------------------------------------------------------------------------

% Partner Logo
% put the name of the logo image file found in graphics path
\DISPartnerLogo{logos/fathom.jpg}


% Title
\DISTitle{Levee Hunter Project Technical Blogpost}

% Draft version:
% If given, adds draft version on front page, a 'DRAFT' box on top of each other page, 
% and line numbers.
% Comment or remove in final version.
\DISVersion{1.0}

% Abstract - % directly after { is important for correct indentation
\DISAbstract{%
  The abstract of my report.
}


% Authors and list of contributors to the analysis
\usepackage{authblk}
\author[a]{Marco Barnfield}
\author[a]{Pawel Mucha}
\author[a]{Liaoshan Li}
\affil[a]{University College London}


% DIS reference code if ever used
\DISRefCode{UCLCDTDIS-2025}

% Author and title for the PDF file
\hypersetup{pdftitle={UCL CDT DIS Document},pdfauthor={The UCL CDT DIS}}

%---------------------------------------------------------------------
% Content
%---------------------------------------------------------------------
\begin{document}

\maketitle

\tableofcontents

\clearpage


%---------------------------------------------------------------------
\newpage
%---------------------------------------------------------------------
\section{Summary}
\label{sec:summary}
In less than 1000 characters (not counting spaces), explain the project goals and outcomes in high level  language accessible to the CEO. Please also include one figure (picture, sketch, diagram, distribution...), which is catchy and easy to explain/understand with a short caption.

%---------------------------------------------------------------------
\newpage
%---------------------------------------------------------------------
\section{What is a levee? And why are we looking for them?}
\label{sec:introduction}
%\input{introduction}
%---------------------------------------------------------------------
-- MB --\\
A levee is a form of flood defence that is used along the banks of a river, typically made from compacted earth and sometimes reinforced with concrete or rocks for additional strength. In the USA, the National Levee Database (NLD) is maintained by the US Army Corps of Engineers (USACE), containing a record of approximately 40,000~km of levee systems, with 36 million people residing in areas protected by these levees.

However, the database is known to be incomplete and contain incorrect coordinates for some levee paths, mostly due to a large portion of the early levee systems which were built by farmers and settlers up to centuries ago. An example of some of these inconsistencies can be found in [FIGURE].

These errors cause issues for Fathom, who need precise levee positions in order to produce highly accurate flood modelling data. These models are used across industries including humanitarian aid, insurance, international development, engineering, conservation, and financial markets.
\section{Image Segmenation}
\label{sec:method}
%\input{method}
%---------------------------------------------------------------------
-- PM --\\
Short overview of image segmentation and how it is used in our project with a short description of Segmenation Models Pytorch and what we have used from there.
\section{Creating a pipeline}
\label{sec:results}
%\input{results}
%---------------------------------------------------------------------
\subsection{Data}
-- LS --\\
Description of downloading and collating the data required from the USGS database, both existing levees and TIF files.
\subsection{Data Processing}
-- PM --\\
Description of mask creation and trimming of larger files.
\subsection{Model Training}
-- MB --\\
Description of GroupedKFold methodology
\subsection{Inference}
-- MB --\\
Description of predictions made and output.
\section{What we learned \& Conclusions}
\label{sec:conclusion}
%\input{conclusion}
%---------------------------------------------------------------------
-- ALL --\\
Couple of short bullet points and a conclusion of the project.
\end{document}
